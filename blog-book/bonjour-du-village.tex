\hypertarget{le-bonjour-du-village}{%
\section{Le bonjour du village !}\label{le-bonjour-du-village}}

\emph{Dimanche 06 mai 2018}

Les premières impressions du Liban (by Flo) :

\begin{itemize}
\tightlist
\item
  il fait chaud
\item
  il y a des voitures partout, notamment des SUV, et ça klaxonne pour
  tout et pour rien
\item
  ça sent la pollution dans la rue
\item
  on s'arrête régulièrement à des checkpoints militaires sur la route,
  plus généralement on note l'omniprésence de militaires
\item
  on se fait réveiller dès 5 heures du matin par des enfants qui jouent
  au foot dans la rue ou par la douce voix de l'épicier d'en face qui
  s'énerve sur tout le monde
\item
  c'est plein d'affiches politiques avec les têtes des candidats (le
  concours "ma binette partout"), ce qui s'explique par la tenue des
  élections législatives aujourd'hui
\end{itemize}

\begin{figure}
\centering
\includegraphics{images/20180506_politique.jpg}
\caption{Un échantillon d'affiches électorales libanaises.}
\end{figure}

Pour ne pas déséquilibrer le tableau, il faut rajouter deux choses : la
nourriture, qui est délicieuse (petit barbecue sur le toit pour nous
recevoir), et l'accueil adorable qui nous est reservé par la famille
d'Elida. De quoi envisager sereinement les prochains jours.

\begin{figure}
\centering
\includegraphics{maps/Liban1.png}
\end{figure}

Nous avons passé deux jours à Beyrouth, ce qui nous a donné
l'opportunité de commencer les visites au pas de course : les grottes de
Jeita, Notre Dame du Liban (NDDL pour ceux qui connaissent), le port de
Byblos.

\begin{figure}
\centering
\includegraphics{images/20180506_Jeita.jpg}
\caption{Les grottes de Jeita, impressionnantes (photo volée exclusive,
bakchiche à la clé pour braver l'interdiction de photographier, on se
fait vite aux coutumes locales ;) )}
\end{figure}

Nous sommes maintenant à Saghbine, un petit village de la Bekaa de
l'ouest, une vallée coincée entre la chaîne montagneuse du mont Liban et
celle de l'anti-Liban (en face, quoi \^{}\^{}). Sur le chemin, nous
avons fait un détour par la ville d'Anjar (voir les photos de ruines
Omeyyades dans le style byzantin ci-dessous).

Ambiance politique aujourd'hui, on reste à l'écart et on vous tient au
courant !

\emph{Florian et Elida}
