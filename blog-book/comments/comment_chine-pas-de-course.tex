\hypertarget{commentaires}{%
\subsection{Commentaires}\label{commentaires}}

\begin{itemize}
\item
  ZRC, \emph{2018-06-19 21h40}

  Deux jours par endroit, c'est rudement rapide ! Content que vous ayez
  aimé la découverte du canard laqué. Vous saurez donc comment manger la
  prochaine fois que vous viendrez dîner...
\item
  Florian LB, \emph{2018-06-24 09h37}

  Salut Ruo ! Je note donc que tu prévois de nous inviter à dîner du
  canard de retour en France, ce sera avec grand plaisir \^{}\^{}\\
  Merci pour tous tes conseils tout au long de notre (court) passage en
  Chine !
\item
  Thibaud, \emph{2018-06-25 20h28}

  C'est marrant que vous vous soyez retrouvés au Red Lantern ! :)
  Récemment on a retrouvé un couple de parisiens qu'on avait croisé
  là-bas et on a reparlé d'un jeune chinois un peu chelou qui trainait
  tous les soirs à l'auberge, et qui s'intéressait beaucoup aux
  programmes des uns et des autres ("Vous partez demain ? Vous allez où
  ?"). Est-ce que vous l'avez rencontré ? :D
\item
  Elida, \emph{2018-07-10 03h49}

  Nop, on n'a pas eu cette chance, mais on a eu le même modèle à
  l'auberge à Luoyang \^{}\^{}
\item
  pythux, \emph{2018-06-25 21h03}

  Très impressionnante cette muraille, ça force le respect... Est-ce que
  vous pouvez vous balader sur n'importe quelle section ? Ou y a-t-il
  une zone délimitée accessible ?
\item
  Elida, \emph{2018-07-10 03h48}

  Je pense qu'il y a des sections trop dangereuses pour y accéder (elle
  est quasiment enfouie à certains endroits). Après y a les sections non
  rénovées comme celle où on est allés, qui méritent amplement
  l'investissement en temps et en énergie pour y arriver. Et puis il y a
  les sections proches de Pékin, toutes rénovées, mais qu'il faut
  partager avec des dizaines de milliers de touristes quotidiens, ce qui
  casse un peu le charme...
\end{itemize}
