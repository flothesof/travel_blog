\hypertarget{commentaires}{%
\subsection{Commentaires}\label{commentaires}}

\begin{itemize}
\item
  ZRC, \emph{2018-07-31 16h13}

  Mais, je n'aperçois pas ces fameux koalas que vous avez vus !\\
  Et je me permets de signaler que la star de la photo 2 n'est pas un
  végétal...
\item
  Florian LB, \emph{2018-08-01 02h24}

  Salut Ruocong ! Quelle lecture attentive ! Concernant les koalas, il
  semblerait qu'Elida prépare un billet spécial sur la faune
  australienne. Peut être y figureront ils ? Quant au végétal qui n'en
  est pas un, je serais ravi que tu éclaire ma lanterne proverbiale.
  C'est quoi ?
\item
  ZRC, \emph{2018-08-01 12h32}

  Il me semble que ce sont des champignons (du règne des Fungi, ou
  Mycètes), non ? Visuellement les recherches semblent pointer vers
  Trametes versicolor.
\item
  Florian LB, \emph{2018-08-02 10h41}

  Ça y ressemble en effet ! Je ne savais pas que les champignons étaient
  un règne à part, voilà pour ma culture générale...
\item
  Thibaud, \emph{2018-08-07 11h16}

  Je dirais que le mont éléphant ressemble à un éléphant, mais ça paraît
  trop simple o\_O\\
  Vous dormiez où sur la route ? En auberge ?
\item
  Florian LB, \emph{2018-08-08 08h01}

  Haha ben on est d'accord avec toi pour l'explication, même si on a pas
  de source plus fiable !\\
  On a pris des locations Airbnb et une auberge de jeunesse sur le
  chemin.
\item
  Franck Mazas, \emph{2018-08-14 10h50}

  La reconversion en coastal engineer me paraît inévitable.
\item
  Florian LB, \emph{2018-08-16 07h35}

  Haha ce n'est pourtant pas la seule possibilité...
\end{itemize}
