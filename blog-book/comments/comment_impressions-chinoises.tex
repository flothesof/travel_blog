\hypertarget{commentaires}{%
\subsection{Commentaires}\label{commentaires}}

\begin{itemize}
\item
  Thibaud, \emph{2018-06-25 20h42}

  Coucou Longhui si tu passes par là :)\\
  La rue des stands de bouffe à Luoyang : miam. Un des rares endroits où
  on avait réussi à comprendre ce qu'on commandait :D Apparemment ça
  s'est mieux passé pour vous !
\item
  Elida, \emph{2018-06-26 00h44}

  Ça commençait bien oui, on a mangé une délicieuse soupe (celle avec le
  pain à émietter dessus), où le serveur se vantait d'être passé sur
  CCTV grâce à ce plat !\\
  Mais tout s'est écroulé au moment où on a commandé une salade de
  fruits sur un stand et que la vendeuse l'a allègrement
  assaisonnée....de mayonnaise ! :'D
\item
  ZRC, \emph{2018-07-02 14h35}

  La "salade" est un exemple typique d'interprétation culinaire. Mon
  grand-père en faisait déjà dans mon enfance en disant que c'était un
  plat occidental.
\item
  Elida, \emph{2018-07-10 03h44}

  Avec la mayo ? 😱
\item
  ZRC, \emph{2018-07-10 07h32}

  Oui, avec une sorte de mayo.
\item
  FalafelDeNataDu75, \emph{2018-07-01 21h23}

  Coucou,

  On m'informe à l'instant que les toilettes à la turque seraient en
  fait originaires de Belgique. Les Turcs n'auraient fait qu'ajouter le
  trou. Bisous.
\item
  Florian LB, \emph{2018-07-10 03h49}

  Merci, cher FalafelDeNataDu75. Nous avons soumis ton information au
  guide volontaire du musée d'Edo Tokyo, qui était persuadé jusque là
  que c'était les japonais qui avaient inventé les toilettes à la
  turque. Il nous a promis de mener des nouvelles recherches pour
  élucider ce mystère !
\end{itemize}
