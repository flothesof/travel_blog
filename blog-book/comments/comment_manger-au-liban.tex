\hypertarget{commentaires}{%
\subsection{Commentaires}\label{commentaires}}

\begin{itemize}
\item
  Thibaud, \emph{2018-05-24 16h06}

  C'est le genre d'article qui me donne envie de partir :D\\
  Merci pour les légendes ! N'oubliez pas quand même qu'on a des fraises
  et des pommes chez nous. On dirait la documentation d'Alex sur un
  constructeur : "// Constructor."\\
  (Ceci est un test pour voir si l'intéressé parcourt votre blog :p)
\item
  Florian LB, \emph{2018-05-26 09h55}

  Haha c'est le genre de réaction qu'on espérait provoquer ;)\\
  On oublie pas qu'on a des pommes et des fraises en France, c'est juste
  qu'on n'arrive pas à ne pas être exhaustifs dans notre collecte de
  photos \^{}\^{}\\
  Pour l'instant je n'ai pas encore eu de commentaire de la part d'Alex,
  mais j'imagine que ça ne va pas tarder...
\item
  kje, \emph{2018-05-28 21h09}

  Du coup, une préférence entre le taboulé de la cantine 3 et celui que
  tu manges au Liban ?
\item
  Florian LB, \emph{2018-06-01 08h51}

  Rien à voir ! Mais j'imagine que tu connaissais déjà la réponse à
  cette question... Pour ceux qui n'ont pas été confronté à cette
  expérience : "taboulé" est un terme qui recouvre un grand nombre de
  plats très différents les uns des autres. Il faudrait sans doute
  exiger un adjectif obligatoire quand on l'utilise pour éviter les
  ambiguïtés.
\item
  pythux, \emph{2018-05-31 10h15}

  Excellent Flo ! Ça fait saliver tout ça :) Ça me rappel le classique
  "dîner chez les potes indiens" : "Allez on passe à table ?"; "Quoi ?
  Je croyais qu'on avait fini là", et "hé non, c'était l'apéritif !"...
  Je me fait avoir à chaque fois.
\item
  Florian LB, \emph{2018-06-01 08h53}

  Je ne savais pas que la gastronomie indienne avait ce point commun
  avec la libanaise. C'est ça de faire des plats apéritifs trop bons...
\item
  Timothée Nicolas, \emph{2018-06-03 20h56}

  Omg ça a l'air tellement bon ! C'était bon la barbaque crue ? J'avais
  toujours eu du mal à imaginer quand Farah disait qu'elle bouffait du
  foie d'agneau cru au petit déj. T'as goûté ? C'est bon ?
\item
  Florian LB, \emph{2018-06-04 06h56}

  J'ai seulement goûté et il faut dire que cela a un goût assez fort.
  Donc je n'irais pas jusqu'à dire que j'ai apprécié. Heureusement que
  ce n'était pas au petit déjeuner mais au déjeuner chez nous...
\end{itemize}
