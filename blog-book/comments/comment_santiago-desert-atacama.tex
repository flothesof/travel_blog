\hypertarget{commentaires}{%
\subsection{Commentaires}\label{commentaires}}

\begin{itemize}
\item
  Thibaud, \emph{2018-09-16 08h55}

  Super belles ces photos en effet 😍\\
  Les salars et la vallée de la Lune... je prends mon billet 😜
\item
  Florian LB, \emph{2018-09-16 19h40}

  Haha si il te reste des vacances à prendre, ça vaut le coup en effet !
  La vallée de la Lune c'est vraiment très beau...
\item
  pythux, \emph{2018-10-09 19h39}

  La vallée de la Lune, vous y êtes allé avec la Polo ?
\item
  Florian LB, \emph{2018-10-12 19h53}

  Oui, mais attention dans les virages, car ça turne ! (je crois que
  c'est copyright Franck mais bon)
\item
  ZRC, \emph{2018-09-17 10h47}

  Malheureux, évitez de dire "pacifier la région des rebelles indigènes"
  à de vrais Chiliens ! C'est plutôt "envahir la région et dominer les
  peuples indigènes"... D'ailleurs, par rapport à notre discussion sur
  l'origine du merquén/merken, ça vient en fait du mapuche, pas des
  colons allemands.

  Photo 48 : Whooaaa\\
  Photo 52 : Je dis vicuña ! Je crois que c'est que j'avais vu aussi sur
  les hauts plateaux et l'alpaca est plutôt laineux.\\
  Et je me permets d'insister, le Sud du pays vaut tout autant le détour
  !
\item
  Florian LB, \emph{2018-09-18 12h49}

  Effectivement le mot pacifier vient directement du point de vue des
  conquérants, ce qui n'était pas forcément notre intention. L'histoire
  est malheureusement écrite par le vainqueur...\\
  Tu remportes le premier prix de notre challenge Llama pour ton
  identification réussie !
\item
  pythux, \emph{2018-10-09 19h38}

  En même temps, on leur avait bien dit qu'il fallait pas s'y fier...
\end{itemize}
