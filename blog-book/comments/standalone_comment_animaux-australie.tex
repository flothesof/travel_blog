\PassOptionsToPackage{unicode=true}{hyperref} % options for packages loaded elsewhere
\PassOptionsToPackage{hyphens}{url}
%
\documentclass[]{article}
\usepackage{lmodern}
\usepackage{amssymb,amsmath}
\usepackage{ifxetex,ifluatex}
\usepackage{fixltx2e} % provides \textsubscript
\ifnum 0\ifxetex 1\fi\ifluatex 1\fi=0 % if pdftex
  \usepackage[T1]{fontenc}
  \usepackage[utf8]{inputenc}
  \usepackage{textcomp} % provides euro and other symbols
\else % if luatex or xelatex
  \usepackage{unicode-math}
  \defaultfontfeatures{Ligatures=TeX,Scale=MatchLowercase}
\fi
% use upquote if available, for straight quotes in verbatim environments
\IfFileExists{upquote.sty}{\usepackage{upquote}}{}
% use microtype if available
\IfFileExists{microtype.sty}{%
\usepackage[]{microtype}
\UseMicrotypeSet[protrusion]{basicmath} % disable protrusion for tt fonts
}{}
\IfFileExists{parskip.sty}{%
\usepackage{parskip}
}{% else
\setlength{\parindent}{0pt}
\setlength{\parskip}{6pt plus 2pt minus 1pt}
}
\usepackage{hyperref}
\hypersetup{
            pdfborder={0 0 0},
            breaklinks=true}
\urlstyle{same}  % don't use monospace font for urls
\setlength{\emergencystretch}{3em}  % prevent overfull lines
\providecommand{\tightlist}{%
  \setlength{\itemsep}{0pt}\setlength{\parskip}{0pt}}
\setcounter{secnumdepth}{0}
% Redefines (sub)paragraphs to behave more like sections
\ifx\paragraph\undefined\else
\let\oldparagraph\paragraph
\renewcommand{\paragraph}[1]{\oldparagraph{#1}\mbox{}}
\fi
\ifx\subparagraph\undefined\else
\let\oldsubparagraph\subparagraph
\renewcommand{\subparagraph}[1]{\oldsubparagraph{#1}\mbox{}}
\fi

% set default figure placement to htbp
\makeatletter
\def\fps@figure{htbp}
\makeatother


\date{}

\begin{document}

\hypertarget{commentaires}{%
\subsection{Commentaires}\label{commentaires}}

\begin{itemize}
\item
  Franck Mazas, \emph{2018-08-14 10h47}

  Emu, en bon français, c'est plutôt émeu. Mettons ça sur le compte de
  l'émotion !\\
  Je suggérerais bien le développement d'un script Python pour la
  reconnaissance automatique des volatiles à partir d'une base de
  données, mais j'aurais trop peur que les pythons bouffent les oiseaux,
  c'est bien connu.
\item
  Florian LB, \emph{2018-08-16 07h39}

  Aaaah c'est pour ce genre de précision qu'on apprécie tes commentaires
  Franck ! Merci, je vais essayer de corriger ça. Quant au script
  python, il va devoir attendre je suis encore en vacances pour 3 mois
  :)
\item
  Didier VEZINET, \emph{2018-08-29 20h17}

  Le truc ou tu sêches c'est pas une variété de héron ?\\
  Ils sont trop mignons les wallabies :-)
\item
  Florian LB, \emph{2018-09-02 00h07}

  Des hérons... possible !\\
  C'est sûr que les wallabies et les kangourous, on les aurait bien
  approchés de plus près. Mais ils sont trop timides !
\end{itemize}

\end{document}
