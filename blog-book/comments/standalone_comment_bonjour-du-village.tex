\PassOptionsToPackage{unicode=true}{hyperref} % options for packages loaded elsewhere
\PassOptionsToPackage{hyphens}{url}
%
\documentclass[]{article}
\usepackage{lmodern}
\usepackage{amssymb,amsmath}
\usepackage{ifxetex,ifluatex}
\usepackage{fixltx2e} % provides \textsubscript
\ifnum 0\ifxetex 1\fi\ifluatex 1\fi=0 % if pdftex
  \usepackage[T1]{fontenc}
  \usepackage[utf8]{inputenc}
  \usepackage{textcomp} % provides euro and other symbols
\else % if luatex or xelatex
  \usepackage{unicode-math}
  \defaultfontfeatures{Ligatures=TeX,Scale=MatchLowercase}
\fi
% use upquote if available, for straight quotes in verbatim environments
\IfFileExists{upquote.sty}{\usepackage{upquote}}{}
% use microtype if available
\IfFileExists{microtype.sty}{%
\usepackage[]{microtype}
\UseMicrotypeSet[protrusion]{basicmath} % disable protrusion for tt fonts
}{}
\IfFileExists{parskip.sty}{%
\usepackage{parskip}
}{% else
\setlength{\parindent}{0pt}
\setlength{\parskip}{6pt plus 2pt minus 1pt}
}
\usepackage{hyperref}
\hypersetup{
            pdfborder={0 0 0},
            breaklinks=true}
\urlstyle{same}  % don't use monospace font for urls
\setlength{\emergencystretch}{3em}  % prevent overfull lines
\providecommand{\tightlist}{%
  \setlength{\itemsep}{0pt}\setlength{\parskip}{0pt}}
\setcounter{secnumdepth}{0}
% Redefines (sub)paragraphs to behave more like sections
\ifx\paragraph\undefined\else
\let\oldparagraph\paragraph
\renewcommand{\paragraph}[1]{\oldparagraph{#1}\mbox{}}
\fi
\ifx\subparagraph\undefined\else
\let\oldsubparagraph\subparagraph
\renewcommand{\subparagraph}[1]{\oldsubparagraph{#1}\mbox{}}
\fi

% set default figure placement to htbp
\makeatletter
\def\fps@figure{htbp}
\makeatother


\date{}

\begin{document}

\hypertarget{commentaires}{%
\subsection{Commentaires}\label{commentaires}}

\begin{itemize}
\item
  François Baqué, \emph{2018-05-08 06h57}

  Bonjour depuis la Provence où la météo reste bien variable et
  fraiche... mois de mai hyper calme avec tous nos ponts et viaducs...\\
  Merci pour votre reportage libanais, au pays des cèdres majestueux et
  millénaires.\\
  Irez-vous à Byblos ? Ruines entremêlées sur 2000 ans d'histoire : des
  phéniciens aux croisés puis aux temps modernes !\\
  Vous êtes au cœur de l'origine de notre civilisation méditerranéenne :
  le barycentre (on reste dans les maths ?) de la culture mondiale
  actuelle (hors Chine : chaque chose en son temps).\\
  A très bientôt.\\
  François
\item
  Florian LB, \emph{2018-05-09 09h43}

  Bonjour François, merci pour ton message. Nous sommes passés en coup
  de vent à Byblos, mais nous y retournerons sans doute. Effectivement,
  les traces historiques sont très nombreuses au Liban et elles
  s'étalent sur des centaines et des centaines d'années, à ne plus
  savoir quelle époque on est en train de regarder. Concernant la météo,
  elle s'est nettement rafraîchie dans la vallée de la Bekaa depuis les
  lignes écrites ci-dessus...\\
  A bientôt,\\
  Florian
\end{itemize}

\end{document}
