\PassOptionsToPackage{unicode=true}{hyperref} % options for packages loaded elsewhere
\PassOptionsToPackage{hyphens}{url}
%
\documentclass[]{article}
\usepackage{lmodern}
\usepackage{amssymb,amsmath}
\usepackage{ifxetex,ifluatex}
\usepackage{fixltx2e} % provides \textsubscript
\ifnum 0\ifxetex 1\fi\ifluatex 1\fi=0 % if pdftex
  \usepackage[T1]{fontenc}
  \usepackage[utf8]{inputenc}
  \usepackage{textcomp} % provides euro and other symbols
\else % if luatex or xelatex
  \usepackage{unicode-math}
  \defaultfontfeatures{Ligatures=TeX,Scale=MatchLowercase}
\fi
% use upquote if available, for straight quotes in verbatim environments
\IfFileExists{upquote.sty}{\usepackage{upquote}}{}
% use microtype if available
\IfFileExists{microtype.sty}{%
\usepackage[]{microtype}
\UseMicrotypeSet[protrusion]{basicmath} % disable protrusion for tt fonts
}{}
\IfFileExists{parskip.sty}{%
\usepackage{parskip}
}{% else
\setlength{\parindent}{0pt}
\setlength{\parskip}{6pt plus 2pt minus 1pt}
}
\usepackage{hyperref}
\hypersetup{
            pdfborder={0 0 0},
            breaklinks=true}
\urlstyle{same}  % don't use monospace font for urls
\setlength{\emergencystretch}{3em}  % prevent overfull lines
\providecommand{\tightlist}{%
  \setlength{\itemsep}{0pt}\setlength{\parskip}{0pt}}
\setcounter{secnumdepth}{0}
% Redefines (sub)paragraphs to behave more like sections
\ifx\paragraph\undefined\else
\let\oldparagraph\paragraph
\renewcommand{\paragraph}[1]{\oldparagraph{#1}\mbox{}}
\fi
\ifx\subparagraph\undefined\else
\let\oldsubparagraph\subparagraph
\renewcommand{\subparagraph}[1]{\oldsubparagraph{#1}\mbox{}}
\fi

% set default figure placement to htbp
\makeatletter
\def\fps@figure{htbp}
\makeatother


\date{}

\begin{document}

\hypertarget{commentaires}{%
\subsection{Commentaires}\label{commentaires}}

\begin{itemize}
\item
  ZRC, \emph{2018-06-03 15h48}

  Moi je dis que la statue de la photo 23 ne repose pas exactement sur
  l'eau... Flo aurait pu citer un texte qu'on a vu en cours de russe,
  qui disait que la vie était dure et qu'on ne souriait pas pour rien en
  Russie.
\item
  Thibaud, \emph{2018-06-03 17h59}

  C'est vrai que la photo 23 est chelou.\\
  J'avais lu à peu près la même chose sur les sourires en Russie : on ne
  sourit pas juste par politesse, ça peut d'ailleurs être considéré
  comme offensant !
\item
  Florian LB, \emph{2018-06-03 21h04}

  Je ne savais pas tout ça. Du coup, je viens de taper la requête dans
  mon moteur de recherche favori et il semble que le sourire russe soit
  un point d'interrogation pour beaucoup de gens. Me voilà rassuré. :-)
\item
  Florian LB, \emph{2018-06-03 20h52}

  Ah zut, je ne m'en souviens pas ! Tu crois que tu peux retrouver le
  texte et le mettre ici ?

  En tout cas bravo, tu as trouvé la photo truquée ! On a utilisé un
  écran de smartphone pour donner un reflet de ciel à l'image.
\item
  Timothée Nicolas, \emph{2018-06-03 21h05}

  C'est magnifique et très intéressant ! On a hâte d'en savoir plus !
\end{itemize}

\end{document}
