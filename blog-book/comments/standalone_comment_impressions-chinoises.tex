\PassOptionsToPackage{unicode=true}{hyperref} % options for packages loaded elsewhere
\PassOptionsToPackage{hyphens}{url}
%
\documentclass[]{article}
\usepackage{lmodern}
\usepackage{amssymb,amsmath}
\usepackage{ifxetex,ifluatex}
\usepackage{fixltx2e} % provides \textsubscript
\ifnum 0\ifxetex 1\fi\ifluatex 1\fi=0 % if pdftex
  \usepackage[T1]{fontenc}
  \usepackage[utf8]{inputenc}
  \usepackage{textcomp} % provides euro and other symbols
\else % if luatex or xelatex
  \usepackage{unicode-math}
  \defaultfontfeatures{Ligatures=TeX,Scale=MatchLowercase}
\fi
% use upquote if available, for straight quotes in verbatim environments
\IfFileExists{upquote.sty}{\usepackage{upquote}}{}
% use microtype if available
\IfFileExists{microtype.sty}{%
\usepackage[]{microtype}
\UseMicrotypeSet[protrusion]{basicmath} % disable protrusion for tt fonts
}{}
\IfFileExists{parskip.sty}{%
\usepackage{parskip}
}{% else
\setlength{\parindent}{0pt}
\setlength{\parskip}{6pt plus 2pt minus 1pt}
}
\usepackage{hyperref}
\hypersetup{
            pdfborder={0 0 0},
            breaklinks=true}
\urlstyle{same}  % don't use monospace font for urls
\setlength{\emergencystretch}{3em}  % prevent overfull lines
\providecommand{\tightlist}{%
  \setlength{\itemsep}{0pt}\setlength{\parskip}{0pt}}
\setcounter{secnumdepth}{0}
% Redefines (sub)paragraphs to behave more like sections
\ifx\paragraph\undefined\else
\let\oldparagraph\paragraph
\renewcommand{\paragraph}[1]{\oldparagraph{#1}\mbox{}}
\fi
\ifx\subparagraph\undefined\else
\let\oldsubparagraph\subparagraph
\renewcommand{\subparagraph}[1]{\oldsubparagraph{#1}\mbox{}}
\fi

% set default figure placement to htbp
\makeatletter
\def\fps@figure{htbp}
\makeatother


\date{}

\begin{document}

\hypertarget{commentaires}{%
\subsection{Commentaires}\label{commentaires}}

\begin{itemize}
\item
  Thibaud, \emph{2018-06-25 20h42}

  Coucou Longhui si tu passes par là :)\\
  La rue des stands de bouffe à Luoyang : miam. Un des rares endroits où
  on avait réussi à comprendre ce qu'on commandait :D Apparemment ça
  s'est mieux passé pour vous !
\item
  Elida, \emph{2018-06-26 00h44}

  Ça commençait bien oui, on a mangé une délicieuse soupe (celle avec le
  pain à émietter dessus), où le serveur se vantait d'être passé sur
  CCTV grâce à ce plat !\\
  Mais tout s'est écroulé au moment où on a commandé une salade de
  fruits sur un stand et que la vendeuse l'a allègrement
  assaisonnée....de mayonnaise ! :'D
\item
  ZRC, \emph{2018-07-02 14h35}

  La "salade" est un exemple typique d'interprétation culinaire. Mon
  grand-père en faisait déjà dans mon enfance en disant que c'était un
  plat occidental.
\item
  Elida, \emph{2018-07-10 03h44}

  Avec la mayo ? 😱
\item
  ZRC, \emph{2018-07-10 07h32}

  Oui, avec une sorte de mayo.
\item
  FalafelDeNataDu75, \emph{2018-07-01 21h23}

  Coucou,

  On m'informe à l'instant que les toilettes à la turque seraient en
  fait originaires de Belgique. Les Turcs n'auraient fait qu'ajouter le
  trou. Bisous.
\item
  Florian LB, \emph{2018-07-10 03h49}

  Merci, cher FalafelDeNataDu75. Nous avons soumis ton information au
  guide volontaire du musée d'Edo Tokyo, qui était persuadé jusque là
  que c'était les japonais qui avaient inventé les toilettes à la
  turque. Il nous a promis de mener des nouvelles recherches pour
  élucider ce mystère !
\end{itemize}

\end{document}
