\PassOptionsToPackage{unicode=true}{hyperref} % options for packages loaded elsewhere
\PassOptionsToPackage{hyphens}{url}
%
\documentclass[]{article}
\usepackage{lmodern}
\usepackage{amssymb,amsmath}
\usepackage{ifxetex,ifluatex}
\usepackage{fixltx2e} % provides \textsubscript
\ifnum 0\ifxetex 1\fi\ifluatex 1\fi=0 % if pdftex
  \usepackage[T1]{fontenc}
  \usepackage[utf8]{inputenc}
  \usepackage{textcomp} % provides euro and other symbols
\else % if luatex or xelatex
  \usepackage{unicode-math}
  \defaultfontfeatures{Ligatures=TeX,Scale=MatchLowercase}
\fi
% use upquote if available, for straight quotes in verbatim environments
\IfFileExists{upquote.sty}{\usepackage{upquote}}{}
% use microtype if available
\IfFileExists{microtype.sty}{%
\usepackage[]{microtype}
\UseMicrotypeSet[protrusion]{basicmath} % disable protrusion for tt fonts
}{}
\IfFileExists{parskip.sty}{%
\usepackage{parskip}
}{% else
\setlength{\parindent}{0pt}
\setlength{\parskip}{6pt plus 2pt minus 1pt}
}
\usepackage{hyperref}
\hypersetup{
            pdfborder={0 0 0},
            breaklinks=true}
\urlstyle{same}  % don't use monospace font for urls
\setlength{\emergencystretch}{3em}  % prevent overfull lines
\providecommand{\tightlist}{%
  \setlength{\itemsep}{0pt}\setlength{\parskip}{0pt}}
\setcounter{secnumdepth}{0}
% Redefines (sub)paragraphs to behave more like sections
\ifx\paragraph\undefined\else
\let\oldparagraph\paragraph
\renewcommand{\paragraph}[1]{\oldparagraph{#1}\mbox{}}
\fi
\ifx\subparagraph\undefined\else
\let\oldsubparagraph\subparagraph
\renewcommand{\subparagraph}[1]{\oldsubparagraph{#1}\mbox{}}
\fi

% set default figure placement to htbp
\makeatletter
\def\fps@figure{htbp}
\makeatother


\date{}

\begin{document}

\hypertarget{commentaires}{%
\subsection{Commentaires}\label{commentaires}}

\begin{itemize}
\item
  Thibaud, \emph{2018-05-02 16h40}

  Ça a l'air de bien fonctionner mais peut-être que tu auras envie de
  modifier ton CSS pour centrer les photos et les espacer légèrement :D
  Tu pourrais faire un article sur le CSS ensuite :)
\item
  Florian LB, \emph{2018-05-04 08h37}

  Salut Thibaud ! Déjà, merci pour le commentaire. Bravo, tu es le
  premier ! Le CSS, c'est un sujet sensible, mais j'ai demandé à une
  experte de nous prêter main forte pour améliorer ça. L'idée de faire
  un article sur "comment ce blog est fait" me trotte dans la tête, mais
  je vais sans doute attendre encore pas mal de temps avant de l'écrire.
  A bientôt !
\item
  ZRC, \emph{2018-05-03 08h54}

  Tu trouveras peut-être que la photographie, c'est comme l'optimisation
  : quand on optimise dans un ensemble limité de solutions, l'optimum
  est facile à atteindre, mais quand on passe dans un ensemble plus
  vaste et plus proche des optima globaux (un meilleur appareil photo),
  le problème est plus difficile... C'est un équilibre à améliorer entre
  qualité d'approximation et performance d'estimation, quoi.
\item
  Florian LB, \emph{2018-05-04 08h43}

  Salut ZRC. Tu inaugures avec brio les commentaires à base d'analogies
  mathématiques ! Je commence à voir la photographie d'un autre œil en
  tout cas. Maintenant, à l'attaque de ce nouvel espace...
\item
  Michael, \emph{2018-05-29 21h59}

  \href{https://uploads.disquscdn.com/images/d3d52c79db3c778c55746cf8982dd515cea75ba04db4e5912de5a2f20df6195c.png}{https://uploads.disquscdn.c...}
\item
  Florian LB, \emph{2018-06-01 08h55}

  Haha. On attend des commentaires "photos" de ta part sous chaque
  article maintenant ! :-)
\end{itemize}

\end{document}
