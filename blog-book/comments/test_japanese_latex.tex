\documentclass{article}

\usepackage{CJKutf8}
  \usepackage[T1,T2A]{fontenc}
  \usepackage[utf8]{inputenc}

\title{Le livre du tour du monde}
\author{Elida et Florian}

\begin{document}
\maketitle
\tableofcontents

\begin{CJK}{UTF8}{min}
\section{これは最初のセクションである}
日本語で \LaTeX の組版を実証するための導入部分。

フォントはまた、数学的な形態および他の環境で使用することができる
\end{CJK}

\bigskip

Outside the environment Latin characters may be used.

\begin{CJK}{UTF8}{min}
\section{\textnormal{こんにちは}, konnichiwa!}
\end{CJK}

\emph{Lundi 25 juin 2018}

C'est après une très courte nuit dans l'avion qui nous amène de Shanghai
que nous posons pied sur l'archipel japonais, aussi appelé "la banane".
Une fois notre Japan Rail Pass en poche, des retrouvailles nous
attendent à Osaka, où nous sommes accueillis par Sorouch et Wakana.



Après deux courtes journées à Osaka, nous enchaînons avec un séjour dans
l'île la plus à l'ouest de l'archipel nippon, Kyushu.

Ce que nous vous raconterons bientôt, pour la suite de nos aventures au
Japon !

\emph{Florian et Elida}

\end{document}